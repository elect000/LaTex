\documentclass[a4j]{jsarticle}
\title{著作権とソフトウェア}
\author{江畑 拓哉}
\date{6月14日}
\usepackage{ascmac}	% required for `\itembox' (yatex added)
\usepackage{url}
\usepackage{hyperref}
\begin{document}
\maketitle
\part{著作権とは何か}
著作権とは何かという問題を考えたとき、私たちがまず始めに思い浮かべるものは何だろうか。私が思いついたものはフリーソフトだった。最近情報特別演習という科目の際にフリーソフトの規約というものをみる機会があったが、よく考えてみると少し不思議な感覚がする。フリーとは”無い”という意味ではないのだろうか。\par
このような事情と、課題という責務感から著作権についてのレポートをここに作成していく。
\part{著作権という言葉}
私はまず、著作権の大まかな意味をweb上から手に入れることにした。
\begin{itembox}{著作権\cite{def_t}}
 \begin{quote}
 著作権(ちょさくけん、英語: copyright、コピーライト)とは、言語、音楽、絵画、建築、図形、映画、コンピュータプログラムなどの表現形式によって自らの思想・感情を創作的に表現した著作物を排他的に支配する、財産的な権利である。 著作権は特許権や商標権にならぶ知的財産権の一つとして位置づけられている。
  \end{quote}
\end{itembox}
この資料からは私は何となくの形しか見ることができなかったが、コンピュータプログラムから思想や感情が芽生えるとは到底思えないと感じた。また、(日本において)著作権は知的財産権の一つに数え上げられるという記述に違和感を覚えた。確かに知的財産権という名前と関連があるような気がするが、それは著作権の下に知的財産権というものが含まれていると考えてしまっていた。
\begin{center}
 著作権$\supset$知的財産権\\
 $\updownarrow$ \\
 著作権$\subset$知的財産権
\end{center}
\section*{著作権の厳密な定義}
次により深い意味を知るためにまず著作物の厳密な定義を同じくweb上で調べてみた。
\begin{itembox}{著作権法第二条 1\cite{def_t_r}}
 著作物 思想又は感情を創作的に表現したものであつて、文芸、学術、美術又は音楽の範囲に属するものをいう。
\end{itembox}
どうやら先程の説明をより抽象化させただけのような気もしたが、現在の日本の著作権法はこのようになっているので、この通りに理解すればよいのだろう。\par
おそらく著作物というのは絵画や文学や彫刻などの芸術活動によって生み出されたものが示されるものだろう。しかしながら例えばEmacsのロゴなどはどのようになるのだろうか。それは先程触れた、商標権に含まれる、つまり同じ知的財産権に保護されるものであるものの、別のカテゴリに含まれているということであるそうだ。\par
著作権はこの著作物を保護するものであるのは当然であるとして、実際にどのように保護しているのか、これを今後調べていく課題とする。
\part{日本の著作権}
さて、調べた結果国々で様々な著作権法があることが分かった。しかしながら、私には日本の著作権が最も関連があるだろうから、日本の著作権について論じていこうと思う。
\section*{現在の法律}
現在日本では、著作権法という法律が著作物を守っている。その具体的内容は、実際の法律( \url{ http://law.e-gov.go.jp/htmldata/S45/S45HO048.html } )を参照いただきたい。\par
大まかには、全体にかかる規則、権利の具体的内容、出版権、著作隣接権、などの権利の保障だけでなく、権利侵害された場合についても詳細に書かれている。おそらくこのレポートを百程積み上げなければこれには届かないだろう。
\subsection*{著作権の対象}
では、このことから著作権の対象についてより理解を含めていきたい。\par
例えば、私が何かしらのゲームなりソフトウェアをインターネット上に無料で公開した場合、原則として著作権が発生する。当然不適切な利用があれば訴えることも可能である。\par
ところがもし私がこれをオープンンソースとして公開したならばどうだろう。\par
しかしオープンソースは著作権を放棄したわけではないので、もし利用規約に違反すれば違反として処理される。\cite{def_oss}
もしどうしても私が著作権を放棄した上で公開したいと思ったなら、パブリックドメインソフトウェアとして公開しなければならない。 \cite{def_pds} \par
 とはいっても著作権に注意するためフリーの素材を使いましょう、と中学校や高校では教えられるため、勘違いをしてフリーソフトには著作権がないと錯覚してしまうこともあるのだろうと予想している。\par
\begin{center}
 \begin{tabular}{|c|c|c|} %http://www.itpassportsiken.com/bbs/0705.html
  \hline
 &オープンソースソフトウェア &パブリックドメインドフトウェア \\ \hline
 著作権&あり &なし \\  \hline
 利用規約&あり &なし \\ \hline
 二次創作&著作権なし &著作権あり \\ \hline
 \end{tabular} \cite{def_diff}
 \end{center}
\subsection*{著作権の期間}
さて、ここでは今まで理解してきた著作権が、どのくらいまで維持されるのかという問題について調べていく。至極わかりやく一言で言い表すのならば、”基本四十年、映画は七十年”である。\cite{def_time} \par
ここでややこしくなるのが、ソフトウェアの”特許権”である。これは二十年となっており、具体的内容は、真似したわけではないが似てしまったソースコードも特許権の侵害とするということなのだそうだ。 \cite{def_tokkyo} \par
ここで再び繰り返すことになるかもしれないが、ソフトウェアの”著作権”は四十年だ。
\subsection*{著作権の対象外}
次に著作権から免れる事例を幾つか紹介しよう。 \par
まずは教育活動目的での使用である。教師や生徒が節度を守って学習のために用いる分には問題がないということらしい。\par
節度というものは非常に重要で、たとえばレポートにインターネット上のレポートをコピー&ペーストをすれば、単位が貰えないどことか、教務室で長時間に及ぶ説法が待っている。\par
次に先程紹介したパブリックドメインソフトウェアである。これは著作権を放棄しているため、勿論のこと著作権の対象外。\par
他にも様々な種類があるのだろうが、ここで重要なことは、”大抵のものには著作権がある”ということである。
\section*{著作権の歴史}
著作権の歴史の始まりは、印刷技術や識字率の向上に伴った、印刷技術の複製権(独占権)なのだそうだ。場所はイギリスで、1662年のことだ。\par
正確な著作権法は、1545年に制定されたヴェネツィアで制定されたそうだ。 \par
そこから現在、世界的な著作権の条約としてベルヌ条約が結ばれている。\par
しかしながら、ソフトウェアなどの著作権にはまだ問題が多く残っているらしい。\cite{def_his} 
\subsection*{海外の著作権}
海外の著作権として、ここでは著作権発祥の地、イギリスの法律を紹介したい。しかしながら、306条もあるため、ここではソフトウェアの著作権について調べる。
\begin{verbatim}
    イギリス著作権法(1992年12月改正、1993年1月施行)
第29条(研究及び私的学習)
(1)	研究又は私的学習を目的とする文芸、演劇、音楽又は美術の著作物の公正利用は、著作物の、又は発行された版の場合には印刷配列の、いずれの著作権をも侵害しない。
(2)	(略)
(3)	(略)
(4)	次の行為は公正利用ではない。
(a)	低水準の言語で表現されたコンピュータ・プログラムをより高水準の言語で表現されたバージョンに交換すること、又は
(b)	そのようにプログラムを変換する過程に付随して当該プログラムを複製すること。(これらの行為は第50条B(逆コンパイル)にしたがって行われるのであれば許容される。)

コンピュータ・プログラム:適法な使用者
第50条A(バックアップ・コピー)
(1)	コンピュータ・プログラムの複製物の適法な使用者が、適法な使用目的のために必要なバックアップ・コピーを作成することは著作権侵害ではない。
(2)	(略)
(3)	本条によってある行為が許容されている場合には、その行為を禁止又は制限するような合意における約定又は条件が存在するか否かは問題とならない(そのような約定は、第296条Aによって無効である)。

第50条B(逆コンパイル)
(1)	低水準の言語で表現されたコンピュータ・プログラムの複製物の適法な使用者が行う次の行為は著作権侵害ではない。
(a)	より高水準の言語で表現されたバージョンに変換すること、又は
(b)	(2)を充たすことを条件として、そのようにプログラムを変換する過程に付随して当該プログラムを複製すること(すなわち、逆コンパイルすること)
(2)	条件は次のとおりである。
(a)	逆コンパイルされるプログラム又は他のプログラムとともに稼働され得る独立したプログラムを創作するために必要な情報を獲得するため、当該プログラムを逆コンパイルすることが必要であること(許容される目的);及び
(b)	そのように獲得された情報が許容される目的以外の目的に使用されないこと。
(3)	特に、次の場合には、(2)の条件は充たされない。
(a)	適法な使用者にとって、許容される目的を達成するために必要な情報があらかじめ利用可能である場合;
(b)	適法な使用者が、逆コンパイルを許容される目的を達成するために必要な行為に限定しない場合;
(c)	適法な使用者が、許容される目的を達成するためには提供する必要のない者に、逆コンパイルによって獲得された情報を提供する場合;又は
(d)	適法な使用者が、逆コンパイルされたプログラムとその表現において実質的に類似したプログラムを作成するため又はその他著作権により制限されている行為を行うために情報を用いる場合。
(4)	本条によってある行為が許容されている場合には、その行為を禁止又は制限するような合意における約定又は条件が存在するか否かは問題とならない(そのような約定は、第296条Aによって無効である)。

第50条C(適法な使用者に許容されるその他の行為)
(1)	コンピュータ・プログラムの複製物の適法な使用者が行う複製又は翻案は、それが次のようなものであることを条件に、著作権侵害ではない。
(a)	適法な使用のために必要であること;及び、
(b)	使用が適法であると規定する合意における約定又は条件において禁止されていないこと。
(2)	特に、エラーの修正のために複製又は翻案することは、コンピュータ・プログラムの適法な使用のために必要である。
(3)	本条は第50条A又は第50条Bにより許容された複製又は翻案には適用されない。

第296条(複製防止を回避するための装置)
(1)	この条は、著作権のある著作物の複製物が、著作権者により又はその許諾を得て、複製防止の電子的形式により公衆に頒布される場合に適用される。
(2)	複製物を公衆に頒布する者は、それが侵害複製物を作成するために使用されることを知り、又はそう信じる理由を有しながら次のことを行う者に対して、著作権者が著作権侵害について有する権利と同一の権利を有する。
(a)	使用された複製防止の形式を回避することを特に予定され、又はそのように適応されたいずれかの装置又は手段を作成し、輸入し、販売若しくは貸与し、販売若しくは貸与のために提供若しくは陳列し、又は販売若しくは貸与のために広告すること。
(b)	ある者がその複製防止の形式を回避することを可能とし、又は援助することを意図される情報を公表すること。
(2A)	(1)における公衆に頒布される複製物がコンピュータ・プログラムの複製物である場合、(2)は、「販売若しくは貸与のために広告すること」という文言が「販売若しくは貸与のために広告すること又は業務上所持(possesses)すること」と読み換えられて適用される。
(3)(略)
(4)この条における複製防止への言及は、著作物の複製を阻止し、若しくは制限し、又は作成された複製物の品質を害することを意図されるいずれかの装置又は手段を含む。

コンピュータ・プログラム
第296条A(一定の約定の無効)
(1)	合意に基づいてコンピュータ・プログラムの使用の権限を有している場合、その合意における約定又は条件は、次の行為を禁止又は制限している限りにおいて無効である。
(a)	合意された使用のために必要であるようなプログラムのバックアップ・コピーを作成すること;
(b)	第50条B(2)の条件が充たされている場合にプログラムの逆コンパイルをすること;又は
(c)	プログラムの要素の基礎にあるアイデア及び原則を理解するためにプログラムの機能を観察、研究又は検査するために装置や手段を使用すること。
(2)	本条においてコンピュータ・プログラムに関する逆コンパイルとは第50条Bにおけると同じ意味である。

\end{verbatim} 
\cite{def_eng}
\part{著作権に関する問題}
今まで著作権について触れてきたが、ここで昨今の著作権に関する問題について取り上げてみよう。
\subsection*{保護期間の問題}
まず最も最近世間を大きく騒がせたTPPに関する問題だ。今回のTPP締結に関し著作権の保護期間が一律70年となった。つまり、もし私がソフトウェアを作成したならば、それにおける著作権は70年になるということらしい。\cite{def_re}
\subsection*{利用者の認識の問題}
そして、利用者の認識の問題がある。過去の論文がコピーされる事件は私達の耳に新しい事だろうが、それ以前に冒頭に述べたような 、フリー=無し、という感覚が身に付いてしまっている。また最近は無料で著作物を閲覧できることから、どれが著作権があるか、あるいは”フリー”なのか、著作権がないのか、この境界が甘くなっているような感覚がしているのだ。
\part{これからの著作権}
<<<<<<< HEAD
これからは著作権は厳しくなるべきという流れでもあるのか、規制を厳しくすることにばかり気を配って入るように感じる。しかしながら、IT産業に強いアメリカでは著作権が厳しい一方でも、研究機関の成果のソフトウェアのオープンソース化の運動が盛んであったり、著作者への配慮に関する法律も充実している。日本も、厳しい法律を作ることに固執するだけでなく、その認知度を高めたり、より合理的な法律に改めていく必要があるように感じる。\cite{def_m_o}、
=======
これからは著作権は厳しくなるべきという流れでもあるのか、規制を厳しくすることにばかり気を配って入るように感じる。しかしながら、IT産業に強いアメリカでは著作権が厳しい一方でも、研究機関の成果のソフトウェアのオープンソース化の運動が盛んであったり、著作者に目を向けた法律を充実させており、日本の、厳しくすれば良い法律、という考え方とは程遠い。\cite{def_m_o} \par
私は、今後の法律は法律家や政治家だけでなく、私たち自身も、より著作権に対して理解を含めていく必要があるのだろうと思う。
>>>>>>> origin/master

\bibliographystyle{junsrt}
\bibliography{extra_3}
\end{document}
%http://www.cas.go.jp/jp/tpp/torikumi/index.html#seibihouan
%http://www.huffingtonpost.jp/taro-yamada/tpp-copyright-law_b_9509084.html
%