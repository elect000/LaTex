\documentclass[a4j]{jsarticle}
\title{著作権とソフトウェア}
\author{江畑 拓哉}
\date{6月14日}
\usepackage{ascmac}	% required for `\itembox' (yatex added)
\usepackage{url}
\begin{document}
\maketitle
\part{著作権とは何か}
著作権とは何かという問題を考えたとき、私たちがまず始めに思い浮かべるものは何だろうか。私が思いついたものはフリーソフトだった。最近情報特別演習という科目の際にフリーソフトの規約というものをみる機会があったが、よく考えてみると少し不思議な感覚がする。フリーとは”無い”という意味ではないのだろうか。
このような事情と、課題という責務感から著作権についてのレポートをここに作成していく。
\part{著作権という言葉}
私はまず、著作権の大まかな意味をweb上から手に入れることにした。
\begin{itembox}{著作権\cite{def_t}}
 著作権(ちょさくけん、英語: copyright、コピーライト)とは、言語、音楽、絵画、建築、図形、映画、コンピュータプログラムなどの表現形式によって自らの思想・感情を創作的に表現した著作物を排他的に支配する、財産的な権利である。 著作権は特許権や商標権にならぶ知的財産権の一つとして位置づけられている。
\end{itembox}
この資料からは私は何となくの形しか見ることができなかったが、コンピュータプログラムから思想や感情が芽生えるとは到底思えないと感じた。また、(日本において)著作権は知的財産権の一つに数え上げられるという記述に違和感を覚えた。確かに知的財産権という名前と関連があるような気がするが、それは著作権の下に知的財産権というものが含まれていると考えてしまっていた。
\begin{center}
 著作権$\supset$知的財産権\\
 $\updownarrow$ \\
 著作権$\subset$知的財産権
\end{center}
\section{著作権の厳密な定義}
次により深い意味を知るためにまず著作物の厳密な定義を同じくweb上で調べてみた。
\begin{itembox}{著作権法第二条 1\cite{def_t_r}}
 著作物 思想又は感情を創作的に表現したものであつて、文芸、学術、美術又は音楽の範囲に属するものをいう。
\end{itembox}
どうやら先程の説明をより抽象化させただけのような気もしたが、現在の日本の著作権法はこのようになっているので、この通りに理解すればよいのだろう。\\
おそらく著作物というのは絵画や文学や彫刻などの芸術活動によって生み出されたものが示されるものだろう。しかしながら例えばEmacsのロゴなどはどのようになるのだろうか。それは先程触れた、商標権に含まれる、つまり同じ知的財産権に保護されるものであるものの、別のカテゴリに含まれているということであるそうだ。\\
著作権はこの著作物を保護するものであるのは当然であるとして、実際にどのように保護しているのか、これを今後調べていく課題とする。
\part{日本の著作権}
さて、調べた結果国々で様々な著作権法があることが分かった。しかしながら、私には日本の著作権が最も関連があるだろうから、日本の著作権について論じていこうと思う。
\section{現在の法律}
現在日本では、著作権法という法律が著作物を守っている。その具体的内容は、実際の法律( \url{ http://law.e-gov.go.jp/htmldata/S45/S45HO048.html } )を参照いただきたい。\\
大まかには、全体にかかる規則、権利の具体的内容、出版権、著作隣接権、などの権利の保障だけでなく、権利侵害された場合についても詳細に書かれている。おそらくこのレポートを百程積み上げなければこれには届かないだろう
\subsection*{著作権の対象}
では、このことから著作権の対象についてより理解を含めていきたい。\\
例えば、私が何かしらのゲームなりソフトウェアをインターネット上に無料で公開した場合、原則として著作権が発生する。当然不適切な利用があれば訴えることも可能である。\\
ところがもし私がこれをオープンンソースとして公開したならばどうだろう。\\
しかしオープンソースは著作権を放棄したわけではないので、もし利用規約に違反すれば違反として処理される\cite{def_oss} 
もしどうしても私が著作権を放棄した上で公開したいと思ったなら、パブリックドメインソフトウェアとして公開しなければならない \cite{det_pds} \\
 とはいっても著作権に注意するためフリーの素材を使いましょう、と中学校や高校では教えられるため、勘違いをしてフリーソフトには著作権がないと錯覚してしまうこともあるのだろうと予想している。\\
\begin{center}
 \begin{tabular}{|c|c|c|} %http://www.itpassportsiken.com/bbs/0705.html
 &オープンソースソフトウェア &パブリックドメインドフトウェア \\
 著作権&あり &なし \\
 利用規約&あり &なし \\
 二次創作&著作権なし &著作権あり \\
 \end{tabular} \cite{def_diff}
 \end{center}
\subsection*{著作権の期間}
さて、ここでは
\subsection*{著作権の対象外}
\section{著作権の歴史}
\subsection*{海外の著作権}
\part{著作権に関する問題}
\subsection{保護期間の問題}
\subsection{著作権の範囲}
\subsection{著作権の所持者}
\part{これからの著作権}
\end{document}
