\documentclass[a4j]{jarticle}
\title{心と体の健康}
\author{elect}
\begin{document}
\chapter{心と病とアートの関係}
\section{精神疾患・心理状態の作品の影響(病跡学入門)}
精神症状が題材や表現に結び付く場合
 ルイス・ウェイン
 ムンク
 リチャード・タッド・・・
さまざまな分野の精神状態の表現
 ノイバンシュタイン城
\chapter{ルイス・ウェイン}
(1860年~1939年)
\section{統合失調症と戦ったイラストレーター}
5人の妹がいたが、父親が亡くなり経済難になってしまい、家族を養わなければならなくなった。\\
幸い絵の才能があったため、絵を描いて彼女たちを養った。\\
嫁も迎えることができたが、ガンに侵され結婚後3年で亡くなった。\\
しかしながら、まだ精神を保っていた彼は働いていたが、統合失調症(優しい性格)による低賃金、クライアントに騙しなどから精神の均衡が崩壊\\
晩年は得意な猫のイラストもうまく書けなくなってしまった。\\
\section{統合失調症とは}
思春期・青年期などに起こる
陽性症状や陰性症状がある
\subsection{陽性症状}
幻覚・妄想
\subsection{陰性症状}
思考の貧困・運動減衰・情動鈍麻・意欲欠如

\chapter{エドワードムンク}
初期は、出す作品がことごとく酷評
フランスで印象派の影響を受けた。
帰国後は、作品にデフォルメが目立ち、画風が変わった。
ベルリンに落ち着き、「吸血鬼」、「マドンナ」シリーズなどの絵を仕上げた。
死や病気をテーマにした絵を中心に描く
医者になっていた弟が亡くなったり、妹が統合失調症で入院した
交際相手にピストルを受け、指の怪我を負う
\vec これらのことからアルコール依存の問題が大きくなり、精神病院に入院することになった。
\chapter{リチャード・ダット}
何かに取りつかれたようになり、彼は父親を殺したり、逃亡したフランスで観光客を殺そうとする。
「お伽の木こりの入信の一撃」
\chapter{ルードヴィヒ2世}
建築と音楽に破滅的浪費を繰り返した。「狂王」
多数の凝った城や宮殿を築く「バイエルンのメルヘン王」
普仏戦争で弟オットー1世が精神に異常をきたしたことから、彼は様々なますます不安定に
危惧を感じたことから、家臣は彼を逮捕し、廃位した。
そして、その翌日彼は水死体となって発見された。
\chapter{モーリス・ユトリロ}
壁などの色に用いられた独特の白が印象的であった。
アルコールに溺れていた初期のものの方が一般に評価が高い(後期は色彩の時代)
ユトリロは体が弱く、情緒不安定であったにもかかわらず、育児を彼女の母に放棄。
その母は彼をアルコールで養育。
(父はアルコール依存症)
義父につれられ、精神病院に入院
ユッテルと友人関係を結び、絵を描くことになった。
52歳で絵画収集家の女性と結婚
55歳で母死去
呼吸器疾患で死去
\chapter{佐伯 祐三}
活動期間の大部分をパリで過ごし、フランスで客死した。
芸大で学ぶ
大学時代に家族が結核で亡くなり、自分も大学時代にり患する。
30歳で死去するまで、パリで多くの作品を描いた。
\end{document}