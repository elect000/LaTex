\documentclass[a4j]{jsarticle}
\usepackage{amsmath}

\begin{document}
\section{Hello \AmS-\LaTeX !}
\begin{align}
 \left(
 \begin{array}{cc}
  2&-1 \\
  -3&4 \\
 \end{array}
 \right)
 \left(
 \begin{array}{c}
  1\\
  2\\
 \end{array}
 \right) = \left(
 \begin{array}{c}
  0\\
  5\\
 \end{array}
 \right)
\end{align}
\section{複数行にわたる数式}
\begin{align}
 \overrightarrow{dd} &= \overrightarrow{rd} + \left(\vec{d} \cdot
 \vec{n}\right)\vec{n}\label{eq:vectordd1}\\
 &= \vec{d} - \left(\vec{d} \cdot \vec{n} \right) \vec{n} \label{eq:vectordd2}
\end{align}

 \section{一部の数式にだけ番号を振る}
\begin{align}
 \left(\lambda r.r \right)
 \left(\lambda x. \lambda y.x~y \right)
 & \rightarrow_{\eta} 
 \left(\lambda r.r \right)
 \left(\lambda x.x \right)\nonumber \\
 & \rightarrow_{\beta}
 \lambda x.x
\end{align}

\section{align*環境}
\begin{align*}
 \tau &::= \alpha\, |\, int|\, bool|\, \tau_1 \rightarrow \tau_2 \\
 \sigma &::= \tau\, |\, \forall \alpha. \sigma
\end{align*}

\section{ \textbackslash [コマンドと \textbackslash ]コマンド}
\[
 Fun\left(x, \protect\underbrace{Let%
 \left
 (f, \protect\overbrace{Fun\left(y,x\right)}
 ~{\forall\alpha_2. \alpha_2 \rightarrow \alpha_1},
 f \right)}_%
 {\forall\alpha_2.\alpha_2 \rightarrow \alpha_1}
 \right)
 : \forall\alpha_2.\alpha_1\rightarrow
 \left(
 \alpha_2 \rightarrow \alpha_1
 \right)
\]

\section{インライン数式}
離散フーリエ変換
$X_k \left( k = 0 , 1 , \dots , N - 1 \right)$は級数
$X_k= \sum~{N-1}_{n=0}x_n \mathrm{e}~{-i\frac{2\pi kn}{N}}$となり、
この計算量は$\mathcal{0}\left(N~2\right)$になる。
\end{document}